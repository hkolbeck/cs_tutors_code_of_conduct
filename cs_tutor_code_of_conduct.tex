\documentclass[12pt]{article}

\usepackage{geometry}

\geometry{margin=1in}

\renewcommand{\familydefault}{\sfdefault}
\renewcommand{\maketitle}[1]{%
  \begin{center}%
    {\huge%
    {\sffamily%
      #1
    }}%
    \vspace{2em}
  \end{center}
}


\begin{document}

%\title{Computer Science Tutor Code of Conduct}
%\author{Colleen Murphy}
%\maketitle

\maketitle{Computer Science Tutor Code of Conduct}

\section{Introduction}
The Computer Science department of Portland State University prides 
itself on its excellent tutoring resources for Computer Science 
students. The Computer Science Tutors are known to provide high-quality 
supplemental instruction to students and are highly regarded 
as an invaluable part of PSU's Computer Science education.

As employees of Portland State University, the Computer Science 
Tutors (hereafter known as "the tutors") are expected to follow the
various employee policies of PSU, including the Code of Ethics 
(http://www.pdx.edu/hr/code-ethics), the Professional Standards of 
Conduct \linebreak Policy 
(http://www.pdx.edu/hr/professional-standards-conduct),
and other applicable policies outlined in the Policies section of 
Human Resources' list of Policies, Contracts and Forms \linebreak
(http://www.pdx.edu/hr/policies-contracts-forms\#POLCON). 
Additionally, the tutors are expected to adhere to the policies found 
in this guidebook.

\section{Expectation of personal conduct in the work place}
The tutors are in a unique position in the University such that their work 
place is also a study space, a class space, and a social space. The work 
place shall be defined as the tutor lounge (FAB suite 88),
public IRC channels such as \#cschat, \#cshelp, and \#csXXX class 
channels, and class rooms where CS coursework is being taught. 

The tutors are representative of the CS department and are often new 
students' first view of the department. It is therefore of utmost 
importance that tutors represent themselves and the department well. 

\section{Expectations of the tutors' abilities}
The tutors are expected to have exceptional grasp of material 
from the following courses offered at PSU:

	\begin{itemize}
	\item CS161
  \item CS162
  \item CS163
  \item CS201
  \item CS202
  \item CS250
  \item CS251
	\item CS300
	\item CS311
	\end{itemize}

Exceptions may be made for inability to teach theoretical CS 
(material from CS250, CS251, an CS311) if there is demonstrable 
proficiency in teaching practical	CS (programming). 

Tutors may be hired if they have taken all of the lower division 
coursework and are enrolled in either CS300 or CS311. 

Because the majority of practical (non-theoretical) coursework is 
taught in C++, C, and Java, tutors are expected to be familiar and 
capable in these languages. If any of the suppored courses is taught
in an alternative language (e.g., Python), tutors are still expected
to help students in such courses to the best of their ability. 

The tutors are expected to be comfortable with using Unix-based 
operating systems, especially Solaris and Ubuntu. The tutors should
be comfortable using the command line for programming.

\section{The tutors' responsibilities to the students}
The tutors are responsible for helping students with their CS 
class assignments. This may include:
	\begin{itemize}
	\item Reviewing and clarifying concepts addressed in class or in 
the textbook
	\item Clarifying the meaning of an assignment and making suggestions 
on how to start an assignment
	\item Reviewing a student's code to help diagnose compiler errors or
run-time bugs (acting as a "second pair of eyes")
	\end{itemize}
The tutors are responsible for helping students to use Unix/Linux
as it relates to CS class assignments. This may include:
	\begin{itemize}
	\item Helping students access the CS Solaris system or Linux Lab 
remotely from their personal computers
	\item Helping students navigate the Unix/Linux file system
	\item Helping students with the process of editing, compiling, 
debugging, and executing programs from the command line
	\end{itemize}

\section{The tutors' other responsibilities}
The tutors are responsible for validating students' CS accounts 
and granting CS-level badge access. The tutors are responsible for 
maintaining the two computer labs. This includes:
	\begin{itemize}
	\item ensuring that the printer is stocked with paper and toner and 
is in good working condition
	\item ensuring that students do not unintentionally leave workstations 
locked and thus unavailable for use by other students. Students who need to 
lock a workstation in order to run long-running processes may submit an LRP 
request to the CAT.
	\item ensuring that the lab computers are in working condition and 
reporting broken computers to the CAT
	\item enforcing the food and drink policy
	\item noting when conduct becomes disruptive to other students who are 
using the space to study and either taking action to end the disruption or 
informing the tutor coordinator so that he/she may handle the situation.
	\item If a tutor suspects a student of cheating in a CS class, the 
tutor is expected to inform the student's instructor and should also 
make the situation known to the other tutors and tutor coordinator. 
	\end{itemize}

\section{Things that the tutors are not responsible for}
The tutors are not responsible for helping students with classes that
are not on the list of supported classes in Section 3, Expectations 
of the tutors' abilities. Examples of unsupported classes include 
CS105, CS106, CS classes higher than CS311, and classes outside of 
the CS department. A tutor may help a student with such classwork 
if the tutor possesses such expertise, but must prioritize students 
who need help in supported classes.

The tutors are not responsible for fixing a student's personal computer or 
installing software on a student's personal computer. The exceptions to 
this are that tutors may help students download, install, and use PuTTY,
file transfer applications, and Eclipse, as required by various CS courses.
The tutors are not responsible for helping students install new operating
systems or other software on their personal computers. If a tutor would
like to help a student with such an endeavor, the tutor must make it 
clear to the student that neither the tutors nor the CS department
are responsible for any mishaps that may befall the student's computer.

\section{Activities that are against expectations of tutors}
The following conduct may result in disciplinary action for a tutor:
\begin{itemize}
\item Repeatedly being late or missing shifts or meetings without advance 
notice or any attempt to notify the tutor coordinator of unavoidable 
lateness or absence.
	\begin{itemize}
	\item If a tutor knows in advance that he or she will have to miss
a shift, he or she will attempt to make arrangements with another 
tutor to cover the shift and then notify the tutor coordinator 
of the changes. If no such arrangement can be made, the 
tutor will notify the tutor coordinator of the expected absence and 
reason for missing the shift no later than one week prior to the shift. 
	\item If issues such as illness occur that necessitate a tutor being 
late to or missing a shift without advance planning, the tutor will 
make every attempt to notify the tutor coordinator of the situation
by IRC, email, or phone at least one hour before the shift starts.
\end{itemize}
\item Undermining an instructor's curriculum to tutorees
	\begin{itemize}
	\item Extreme objections to an instructor's curriculum may be addressed 
at tutor meetings or directly with the tutor coordinator.
	\item Confusion or misunderstanding about an instructor's curriculum
can be addressed with other tutors or the tutor coordinator, and may then 
be clarified with the instructor if clarification is warranted.
	\end{itemize}
\item Disrespecting a fellow tutor or anyone else in the work place 
\begin{itemize}
	\item Conflicts between tutors may be mediated by the tutor coordinator.
	\item Tutors should not engage in argumentative or combative discussion 
with members of the community in the work place. 
\end{itemize}
\item Refusing to help a student with legitimate classwork
	\begin{itemize}
	\item If a tutor is truly unable to help a student with legitimate 
classwork from a supported class (see Section 3, Expectations of the tutors' abilities above)
due to inexperience with the particular subject matter, the tutor
may refer the student to another tutor or to the class TA, or encourage the
student to come back during another tutor's shift. 
	\end{itemize}
\item Accepting monetary compensation from students for tutoring during 
their paid tutoring shift
\item Helping a student cheat
\end{itemize}

\section{Conclusion}
This guide may not encapsulate every expectation of the tutors and 
is subject to change. Every new revision will be shared with the 
tutors on staff. 

\end{document}
